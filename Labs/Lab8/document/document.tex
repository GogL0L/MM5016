% Created 2021-05-03 Mon 08:27
% Intended LaTeX compiler: pdflatex
\documentclass[10pt]{article}
     \usepackage{/home/john/texstuff/NoTeX/NotesTeXSW}
     \input{/home/john/skola/test/test3/bold.tex}
     \usepackage{minted}
\date{\today}
\title{MM5016 Laboration 8 Error analysis}
\hypersetup{
 pdfauthor={},
 pdftitle={MM5016 Laboration 8 Error analysis},
 pdfkeywords={},
 pdfsubject={},
 pdfcreator={Emacs 27.2 (Org mode 9.4.4)}, 
 pdflang={Samin}}
\begin{document}

\maketitle

\section{Problems}
\label{sec:org4b758be}

\begin{exercise}[1]  \label{exe:1}
The two quantities \(x\) and \(y\) have a relative error
of \(\epsilon_x\) respectively \(\epsilon_y\). Calculate the error in the following cases:
\begin{enumerate}
\item \(x -2y\)
\item \(4x -5y\)
\item \(3xy\)
\item \(x^{3} / y^{5}\)
\item \(\sqrt{x}\)
\item \(\sqrt{x^{3} }\)
\item \(\sqrt{2xy}\)
\item \(2 \sqrt{\frac{x}{3y}}\)
\item \(2 \sqrt{\frac{x^{3} }{3y}}\)
\item \(\frac{x}{y} + \frac{y}{x}\)
\item \(x^2 + y^2\)
\end{enumerate}
\end{exercise}
\begin{lemma}[error of scaled approximate value] \label{lem:error_of_scaled_approximate_value}
Let \(x\) be a value and \(\tilde{x}\) and approximation with relative error \(\epsilon _x\), and
\(\lambda\) a scalar. Then the relative error of \(\lambda \tilde{x}\) is
\(\epsilon _{\lambda x } = \epsilon _{x}\). This is because \(\lambda \tilde{x} = \lambda x(1+ \epsilon _{\lambda x} )\).
\end{lemma}
\begin{lemma}[error of exponent] \label{lem:error_of_exponent}
Let \(x\) be a value and \(\tilde{x}\) and approximation with relative
error \(\epsilon _{x}\), and \(a\) an integer. Then the relative error of
\(\tilde{x} ^{a}\) is \(\epsilon _{x^{a} } = a \epsilon _{x}\). This is because
\(\epsilon _{x^{a} } = \epsilon _{x x^{a-1} } = \epsilon _{x} + \epsilon _{x^{a-1} }\) .
\end{lemma}
\begin{solution}[1]  \label{sol:1}
(1)
\begin{align*}
\epsilon _{x - 2y} &   = \frac{x}{x - (2y) } \epsilon _{x} -
\frac{(2y) }{x - (2y) } \epsilon _{2y} \\
& = \frac{x}{x - 2 y } \epsilon _{x} -
\frac{2 y }{x - 2 y  } \epsilon _{y}
.
\end{align*}

(2)
\begin{align*}
\epsilon _{4x - 5y}  &  =
\frac{4x}{4x - 5y} \epsilon _{4x} - \frac{5y}{4x - 5y} \epsilon _{5y} \\
& = 
\frac{4x}{4x - 5y} \epsilon _{x} - \frac{5y}{4x - 5y} \epsilon _{y}
.
\end{align*}

(3)
\begin{align*}
\epsilon _{3xy} = \epsilon _{xy} = \epsilon _{x} + \epsilon _{y}
.
\end{align*}

(4)
\begin{align*}
\epsilon _{\frac{x^{3} }{y^{5} } } = \epsilon _{x^{3} } - \epsilon _{y^{5}} = 3 \epsilon _{x} - 5 \epsilon _{y}
.
\end{align*}

(5)
\begin{align*}
 &  \epsilon _{x} = \epsilon _{\sqrt{x}^{2}} = 2 \cdot\epsilon _{\sqrt{x}} \\
\implies & \epsilon _{\sqrt{x}} = \epsilon _{x} / 2
.
\end{align*}

(6)
\begin{align*}
\epsilon _{\sqrt{x^{3}}} = \frac{\epsilon _{x^{3}}}{2} = \frac{3}{2} \epsilon _{x}
.
\end{align*}
(7)
\begin{align*}
\epsilon _{\sqrt{2xy}}  &  = \epsilon _{\sqrt{2} \cdot  \sqrt{x} \cdot \sqrt{y}} \\
& = \epsilon _{\sqrt{x} \cdot  \sqrt{y}} = \epsilon _{\sqrt{x}} + \epsilon _{\sqrt{y}} \\
& = \frac{\epsilon _{x}}{2} + \frac{\epsilon _{y}}{2} 
.
\end{align*}

(7)
\begin{align*}
\epsilon _{2 \sqrt{ \frac{x^{3}}{3y}}}  &  = \epsilon _{\sqrt{ \frac{x^{3}}{3y}}}
= \frac{\epsilon _{\frac{x^{3}}{3y}}}{2} = \frac{\epsilon _{x^{3}} - \epsilon _{3y}}{2}
= \frac{3 \epsilon _{x} - \epsilon _{y}}{2} 
.
\end{align*}
(8)
\begin{align*}
\epsilon _{\frac{x}{y} + \frac{y}{x}}  &
= \frac{\frac{x}{y}}{\frac{x}{y} + \frac{y}{x}} \epsilon _{\frac{x}{y}}
+ \frac{\frac{y}{z}}{\frac{x}{y} + \frac{y}{x}} \epsilon _{\frac{y}{z}}
= \frac{x}{x + \frac{y ^2}{x}} \epsilon _{\frac{x}{y}}
+ \frac{y}{\frac{x}{yz} + \frac{y}{xz}} \epsilon _{\frac{y}{z}} \\
& = 
\frac{x}{x + \frac{y ^2}{x}} \epsilon _{\frac{x}{y}}
+ \frac{y}{\frac{x^2 + y ^2}{xyz}} \epsilon _{\frac{y}{z}} \\
& = 
\frac{x}{x + \frac{y ^2}{x}} \epsilon _{\frac{x}{y}}
+ \frac{x y ^2 z}{x^2 + y ^2} \epsilon _{\frac{y}{z}} \\
& = 
\frac{x ^2}{x ^2 + y ^2} \epsilon _{\frac{x}{y}}
+ \frac{x y ^2 z}{x^2 + y ^2} \epsilon _{\frac{y}{z}} \\
& =
\frac{x}{x ^2 + y ^2} (\epsilon _{\frac{x}{y}} + y ^2 z \epsilon _{\frac{y}{z}}) \\
& =
\frac{x}{x ^2 + y ^2} (\epsilon _{x} - \epsilon _{y} + y ^2 z (\epsilon _{y} - \epsilon _{z}))
.
\end{align*}

(8)
\begin{align*}
\epsilon _{x^2 + y^2}  &   = \frac{x ^2}{x^2 + y ^2} \epsilon _{x ^2} + \frac{y ^2}{x ^2 + y^2} \epsilon _{y ^2} \\
& = \frac{x ^2}{x^2 + y ^2} 2\epsilon _{x} + \frac{y ^2}{x ^2 + y^2} 2\epsilon _{y}
.
\end{align*}
\end{solution}
\begin{exercise}[2]  \label{exe:2}
The radius of a sphere is \(R = (22.2 \pm 0.1)cm\). The radius
of the base of a cylinder is \(r= (12.0 \pm 1.2)cm\), and its
height is \(h=(24.4 \pm 1.1)cm\). What is the total volum
occupied by the sphere and by the cylinder (including the error)?
\end{exercise}
\begin{solution}[2]  \label{sol:2}
Let the subscript \(t\) denote the true value.
\begin{align*}
V _{total}  &  = \frac{4 \pi}{3} (R + \delta R)^{3} + \pi (r+ \delta r) ^2 h \\
& = \frac{4 \pi}{3} (R^{3} + 3 R^2 \delta R + 3 R \delta R ^2 + \delta R^{3})
+ \pi (r ^2 + 2 \delta r+ \delta ^2) (h + \delta h) \\
& = \frac{4 \pi}{3} (R^{3} + 3 R^2 \delta R + 3 R \delta R ^2)
+ \pi (r ^2 + 2 \delta r) (h + \delta h) \\
& = \frac{4 \pi}{3} (R^{3} + 3 R^2 \delta R + 3 R \delta R ^2)
+ \pi (r ^2 h + r ^2 \delta h + 2 \delta h r + 2 \delta r \delta h) \\
& = \frac{4 \pi}{3} (R^{3} + 3 R^2 \delta R)
+ \pi (r ^2 h + r ^2 \delta h + 2r \delta h )
.
\end{align*}

To calculate the minimum volume we let the absolute be so much negative
as possible and the reverse for the maximum, if we plug in these values
we get:
\begin{align*}
55668.1 \leq V _{total} \leq 58068
.
\end{align*}


\end{solution}
\end{document}