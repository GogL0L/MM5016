% Created 2021-04-08 Thu 01:53
% Intended LaTeX compiler: pdflatex
\documentclass[12pt]{article}
\usepackage[utf8]{inputenc}
\usepackage[T1]{fontenc}
\usepackage{graphicx}
\usepackage{geometry}
\usepackage{grffile}
\usepackage{longtable}
\usepackage{wrapfig}
\usepackage{rotating}
\usepackage[normalem]{ulem}
\usepackage{amsmath}
\usepackage{textcomp}
\usepackage{amssymb}
\usepackage{capt-of}
\usepackage[dvipsnames]{xcolor}
\usepackage[colorlinks=true, linkcolor=Blue, citecolor=BrickRed, urlcolor=PineGreen]{hyperref}
\usepackage{indentfirst}
     \geometry{margin=15mm,heightrounded}
     \newtheorem{thm}{Theorem}[section]
     \newtheorem{cor}{Corollary}[thm]
     \newtheorem{lem}{Lemma}[thm]
     \newtheorem{ex}{Example}[ex]
     \usepackage{minted}
\date{\today}
\title{Lab Assignment 2 MM5016}
\hypersetup{
 pdfauthor={},
 pdftitle={Lab Assignment 2 MM5016},
 pdfkeywords={},
 pdfsubject={},
 pdfcreator={Emacs 27.2 (Org mode 9.4.4)}, 
 pdflang={Samin}}
\begin{document}

\maketitle

\section*{Task 1}
\label{sec:org57d1514}

The functions \texttt{find\_root\_newton} and \texttt{find\_root\_secant} in the python file. They give
the right root if the inintal value(s) is (are) inside the interval.

\section*{Task 2}
\label{sec:orgc290f2f}
\subsection*{Finding the roots and testing several initial values}
\label{sec:orgc327088}
Run "python lab2.py" in a terminal to solve the equations with a certain method.
\subsection*{Dependence of initial value}
\label{sec:org37eed1b}
From the lecture notes we now that the error of the iteration
\(x _{n+1}\) to the root \(\alpha\) can be expressed as:
\begin{align*}
\alpha - x _{n+1} = (\alpha - x_n ) ^2 [\frac{-f''(x_n)}{2f'(x_n)}] 
.
\end{align*}

So given that \(\frac{-f''(x_n)}{2f'(x_n)}\) stays roughly the same, the method gets really effective when \(\alpha - x_0 < 1\), as the square will be smaller.


Because the Secant method doesn't use tangents there isn't a simmilar error formula that
can be derived from a taylor expansion. But because secants approximates an interval
better if it's smaller (given that the function is a normal continous function), then such
a choice of initial values will converge faster.
\end{document}