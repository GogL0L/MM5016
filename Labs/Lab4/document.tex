% Created 2021-04-12 Mon 19:08
% Intended LaTeX compiler: pdflatex
\documentclass[10pt]{article}
     \usepackage{/home/john/texstuff/NoTeX/NotesTeXSW}
     \input{/home/john/skola/test/test3/bold.tex}
     \usepackage{minted}
\date{\today}
\title{Lab 4 MM5016}
\hypersetup{
 pdfauthor={},
 pdftitle={Lab 4 MM5016},
 pdfkeywords={},
 pdfsubject={},
 pdfcreator={Emacs 27.2 (Org mode 9.4.4)}, 
 pdflang={Samin}}
\begin{document}

\maketitle


\section{Problems}
\label{sec:orgcbab2ba}

\begin{exercise}[System 1]  \label{exe:System_1}
Use Gaussian elimination to solve the system of linear equations
\begin{align*}
x_1 + 5x_2 = 7 \\
-2x_1 -7x_2 = -5
.
\end{align*}
\end{exercise}
\begin{solution}[]  \label{sol:}
First step is to check that the coefficient of \(x_1\) of the first
equation is non-zero.
Next step is to eliminate all \(x_1\) under line 1. We do this by
adding an appropiate multiple of line 1 to the line we
want to eliminate \(x_1\) from. In this case it means to add the first
line times 2 to second line. That yields the following system:
\begin{align*}
x_1 + 5x_2 = 7 \\
3x_2 = 9
.
\end{align*}

Our system is now overtriangular, which mean we can go into
the process of solving each equation by themselves by starting
from the bottom and substituting into the equation above
relative of the equation in question.


So we start by solving \(x_2 = 3\) from the second equation. Next step is
to substitute this into the above equation and get the equation
\begin{align*}
x_1 + 15 = 7
.
\end{align*}

And thus we can from this equation solve that \(x_1 = -8\).

\end{solution}
\begin{exercise}[System 2]  \label{exe:System_2}
Use Gaussian elimination to solve the system of linear equation
\begin{align*}
2x_2 + x_3 = -8 \\
x_1 - 2x_2 - 3x_3 = 0 \\
-x_1 + x_2 + 2x_3 = 3
.
\end{align*}
\end{exercise}
\begin{solution}[]  \label{sol:}
First step is to see if the first equation has a non-zero coefficent for
\(x_1\), we see that this is not the case. So we swap this line and the closest
line below which has a non-zero coefficient for \(x_1\). In this case it's
line 2. So we swap line 1 and 2. We get the following system:
\begin{align*}
x_1 - 2x_2 - 3x_3 = 0 \\
2x_2 + x_3 = -8 \\
-x_1 + x_2 + 2x_3 = 3
.
\end{align*}

Next step is to eliminate all \(x_1\) from the equations below by adding
an appropiate multiple of line 1 to it. We see that line 2 already is
free from \(x_2\). So we go to the next line which is line number 3. We see
that we can add 1 times equation 1 to it, which yields the following
system:
\begin{align*}
x_1 - 2x_2 - 3x_3 = 0 \\
2x_2 + x_3 = -8 \\
-x_2 - x_3 = 3
.
\end{align*}

Now we apply Gaussian elimination on this new system of 2 equations and
2 variables under line 1:
\begin{align*}
2x_2 + x_3 = -8 \\
-x_2 - x_3 = 3
.
\end{align*}

To solve this system we use the same algorithm as in the previous task, but the
first variable is now \(x_2\) instead. So we eliminate \(x_2\) from the second equation
by adding the first equation times \(\frac{1}{2}\). Which yields the following system:
\begin{align*}
2x_2 + x_3 = -8 \\
- \frac{1}{2} x_3 = 7
.
\end{align*}


\end{solution}
\begin{exercise}[System 3]  \label{exe:System_3}
Use Gaussian elimination to solve the system of linear equations
\begin{align*}
x_1 - 2x_2 -6x_3 = 12 \\
2x_1 + 4x_2 + 12x_3 = -17 \\
x_1 - 4x_2 - 12x_3 = 22
.
\end{align*}
\end{exercise}
\begin{solution}[]  \label{sol:}
As in the previous system we start by eliminating all \(x_1\) from
all the lines below the first line.
To eliminate \(x_1\) from the second line we add \(-2\) times the first line
to it. That yields the following equation:
\begin{align*}
8x_2 + 24x_3 = -41
.
\end{align*}

\end{solution}
\end{document}