% Created 2021-04-12 Mon 21:04
% Intended LaTeX compiler: pdflatex
\documentclass[10pt]{article}
     \usepackage{/home/john/texstuff/NoTeX/NotesTeXSW}
     \input{/home/john/skola/test/test3/bold.tex}
     \usepackage{minted}
\date{\today}
\title{Lab 4 MM5016}
\hypersetup{
 pdfauthor={},
 pdftitle={Lab 4 MM5016},
 pdfkeywords={},
 pdfsubject={},
 pdfcreator={Emacs 27.2 (Org mode 9.4.4)}, 
 pdflang={Samin}}
\begin{document}

\maketitle


\section{Problems}
\label{sec:org0d400bd}

\begin{exercise}[System 1]  \label{exe:System_1}
Use Gaussian elimination to solve the system of linear equations
\begin{align*}
x_1 + 5x_2 = 7 \\
-2x_1 -7x_2 = -5
.
\end{align*}
\end{exercise}
\begin{solution}[]  \label{sol:}
First step is to check that the coefficient of \(x_1\) of the first
equation is non-zero.
Next step is to eliminate all \(x_1\) under row 1. We do this by
adding an appropiate multiple of row 1 to the row we
want to eliminate \(x_1\) from. In this case it means to add the first
row times 2 to second row. That yields the following system:
\begin{align*}
x_1 + 5x_2 = 7 \\
3x_2 = 9
.
\end{align*}

Our system is now overtriangular, which mean we can go into
the process of solving each equation by themselves by starting
from the bottom and substituting into the equation above
relative of the equation in question.


So we start by solving \(x_2 = 3\) from the second equation. Next step is
to substitute this into the above equation and get the equation
\begin{align*}
x_1 + 15 = 7
.
\end{align*}

And thus we can from this equation solve that \(x_1 = -8\).

\end{solution}
\begin{exercise}[System 2]  \label{exe:System_2}
Use Gaussian elimination to solve the system of linear equation
\begin{align*}
2x_2 + x_3 = -8 \\
x_1 - 2x_2 - 3x_3 = 0 \\
-x_1 + x_2 + 2x_3 = 3
.
\end{align*}
\end{exercise}
\begin{solution}[]  \label{sol:}
First step is to see if the first equation has a non-zero coefficent for
\(x_1\), we see that this is not the case. So we swap this row and the closest
row below which has a non-zero coefficient for \(x_1\). In this case it's
row 2. So we swap row 1 and 2. We get the following system:
\begin{align*}
x_1 - 2x_2 - 3x_3 = 0 \\
2x_2 + x_3 = -8 \\
-x_1 + x_2 + 2x_3 = 3
.
\end{align*}

Next step is to eliminate all \(x_1\) from the equations below by adding
an appropiate multiple of row 1 to it. We see that row 2 already is
free from \(x_1\). So we go to the next row which is row number 3. We see
that we can add 1 times equation 1 to it, which yields the following
system:
\begin{align*}
x_1 - 2x_2 - 3x_3 = 0 \\
2x_2 + x_3 = -8 \\
-x_2 - x_3 = 3
.
\end{align*}

Now we apply Gaussian elimination on this new system of 2 equations and
2 variables under row 1:
\begin{align*}
2x_2 + x_3 = -8 \\
-x_2 - x_3 = 3
.
\end{align*}

To solve this system we use the same algorithm as in the previous task, but the
first variable is now \(x_2\) instead. So we eliminate \(x_2\) from the second equation
by adding the first equation times \(\frac{1}{2}\). Which yields the following system:
\begin{align*}
2x_2 + x_3 = -8 \\
- \frac{1}{2} x_3 = -1
.
\end{align*}

So we have the system:
\begin{align*}
x_1 - 2x_2 - 3x_3 = 0 \\
2x_2 + x_3 = -8 \\
-\frac{1}{2}x_3 = -1
.
\end{align*}

So we can start by solving the last equation and then substituting into the above
one and repeat. So from the third equation we get \(x_3 = 2\). Substiuting into the
system we get:
\begin{align*}
x_1 - 2x_2 - 6 = 0 \\
2x_2 + 2 = -8
.
\end{align*}

Solving the last equation we get \(x_2 = -5\). Substituting into the first
equation we get:
\begin{align*}
x_1 + 10 -6 = 0
.
\end{align*}
So we get that \(x_1 = -4\)


\textbf{Final answer:} \((x_1 , x_2 , x_3) = (-4, -5, 2)\).

\end{solution}
\begin{exercise}[System 3]  \label{exe:System_3}
Use Gaussian elimination to solve the system of linear equations
\begin{align*}
x_1 - 2x_2 -6x_3 = 12 \\
2x_1 + 4x_2 + 12x_3 = -17 \\
x_1 - 4x_2 - 12x_3 = 22
.
\end{align*}
\end{exercise}
\begin{solution}[]  \label{sol:}
\(x_1\) is non-zero so we don't need to swap equations as for now.
So we start by eliminating all \(x_1\) from
all the rows below the first row.
To eliminate \(x_1\) from the second row we add \(-2\) times the first row
to it. That yields the following equation:
\begin{align*}
8x_2 + 24x_3 = -41
.
\end{align*}

To eliminate \(x_1\) from the third equation we add \(-1\) times the first equation.
Resulting in the equation:
\begin{align*}
-2x_2 - 6 x_3 = 10
.
\end{align*}

So under row 1 we have the following system of 2 variables and equations:
\begin{align*}
8x_2 + 24x_3 = -41 \\
-2x_2 - 6x_3 = 10
.
\end{align*}

So we start by elimaniting \(x_2\) from the equations under row 1.
We do this by adding \(\frac{1}{4}\) times the first equation to the second
equation. The second equation then becomes:
\begin{align*}
0 = 10 - \frac{41}{4}
.
\end{align*}
Which is a contradiction which means there exists no solutions.

\end{solution}
\end{document}