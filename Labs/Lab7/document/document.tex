% Created 2021-05-01 Sat 12:51
% Intended LaTeX compiler: pdflatex
\documentclass[10pt]{article}
     \usepackage{/home/john/texstuff/NoTeX/NotesTeXSW}
     \input{/home/john/skola/test/test3/bold.tex}
     \usepackage{minted}
\author{John Möller}
\date{\today}
\title{Lab assignment 7 (Iterative method and Gradient descent method)}
\hypersetup{
 pdfauthor={John Möller},
 pdftitle={Lab assignment 7 (Iterative method and Gradient descent method)},
 pdfkeywords={},
 pdfsubject={},
 pdfcreator={Emacs 27.2 (Org mode 9.4.4)}, 
 pdflang={Samin}}
\begin{document}

\maketitle

\section{Task 1}
\label{sec:org2810779}
\begin{exercise}[Task1]  \label{exe:Task1}
Solve the below linear equations using Gaussian elimination.
\begin{align*}
5x_1 - 2x_2 + 3x_3 = -1 \\
-3x_1 + 9x_2 + x_3 = 2 \\
2x_1 - x_2 - 7x_3 = 3
.
\end{align*}

\end{exercise}
\begin{solution}[]  \label{sol:}
Let us first reformulate the equation as an augmented matrix:
\begin{align*}
\left( \begin{array}{c c c | c}
5  &  -2  &  3  &  -1 \\
-3  &  9  &  1  &  2 \\
2  &  -1  &  -7  &  3
\end{array} \right)
.
\end{align*}

Let us first normalise the first row, that yields:

\begin{align*}
\left( \begin{array}{c c c | c}
1  &  -\frac{2}{5}  &  \frac{3}{5}  &  -\frac{1}{5} \\
-3  &  9  &  1  &  2 \\
2  &  -1  &  -7  &  3
\end{array} \right)
.
\end{align*}

Let us now eliminate the second row of the first column by adding
3 times the first row:
\begin{align*}
\left( \begin{array}{c c c | c}
1  &  -\frac{2}{5}  &  \frac{3}{5}  &  -\frac{1}{5} \\
0  &  \frac{39}{5}  &  \frac{14}{5}  &  \frac{7}{5} \\
2  &  -1  &  -7  &  3
\end{array} \right)
.
\end{align*}

Let us now eliminate the third row of the first column by adding
-2 times the first row:

\begin{align*}
\left( \begin{array}{c c c | c}
1  &  -\frac{2}{5}  &  \frac{3}{5}  &  -\frac{1}{5} \\
0  &  \frac{39}{5}  &  \frac{14}{5}  &  \frac{7}{5} \\
0  &  -\frac{1}{5}  &  -\frac{41}{5}  &  \frac{17}{5}
\end{array} \right)
.
\end{align*}

Let us now normalise the second row of the second column by
multiplying with \(\frac{5}{39}\):

\begin{align*}
\left( \begin{array}{c c c | c}
1  &  -\frac{2}{5}  &  \frac{3}{5}  &  -\frac{1}{5} \\
0  &  1  &  \frac{14}{39}  &  \frac{7}{39} \\
0  &  -\frac{1}{5}  &  -\frac{41}{5}  &  \frac{17}{5}
\end{array} \right)
.
\end{align*}

In the normal algorithm we would now add a \(\frac{1}{5}\) times the second row
to the third row, in order to eliminate the third row in the second
column. But we will instead first scale the third row by \(5\), in order
to make it more readable:

\begin{align*}
\left( \begin{array}{c c c | c}
1  &  -\frac{2}{5}  &  \frac{3}{5}  &  -\frac{1}{5} \\
0  &  1  &  \frac{14}{39}  &  \frac{7}{39} \\
0  &  -1 &  -41 &  17
\end{array} \right)
.
\end{align*}
Now we can simply add the second row to the third row in order to eliminate
the third row of the second column:

\begin{align*}
\left( \begin{array}{c c c | c}
1  &  -\frac{2}{5}  &  \frac{3}{5}  &  -\frac{1}{5} \\
0  &  1  &  \frac{14}{39}  &  \frac{7}{39} \\
0  &  0 & \frac{14 - 41 \cdot 39}{39} & \frac{7 + 17 \cdot 39}{39}
\end{array} \right)
.
\end{align*}
From the third row we get that:
\begin{align*} &  \frac{14 -41 \cdot39}{39} x_3 = \frac{7 + 17 \cdot39}{39} \\
\implies &  (14 - 41 \cdot39) x_3 = 7 + 17 \cdot 39 \\
\implies & x_3   = \frac{7 + 17 \cdot39}{14 - 41 \cdot39} \\
&    = \frac{7 + 663}{14 - 1599} \\
 &    = \frac{670}{- 1585} = - \frac{134}{317} 
.
\end{align*}
From the second row of the augmented matrix we get that:
\begin{align*}
 &  x_2 + \frac{14x_3}{39} = \frac{7}{39} \\
\implies & x_2 = \frac{7 - 14x_3}{39} \\
\implies & 39x_2 = 7 - 14x_3 \\
\implies & 39x_2 = 7 + \frac{14 \cdot 134}{317} \\
\implies & 39x_2 = \frac{7 \cdot317 + 14 \cdot 134}{317}  \\
\implies & 39x_2 = \frac{2219  + 1876}{317}  \\
\implies & 39x_2 = \frac{4095}{317}  \\
\implies & 39x_2 = \frac{819 \cdot 5}{317}  &  \text{ we observe that 819 is divisible by 3 }  \\
\implies & 39x_2 = \frac{273 \cdot 3 \cdot 5}{317} &   \text{ 273 is also divisible by 3 } \\
\implies & 39x_2 = \frac{91 \cdot 9 \cdot 5}{317}  &  \text{ 39 is also divisible by 3 } \\
\implies & 13x_2 = \frac{91 \cdot 3 \cdot 5}{317} \\
\implies & x_2 = \frac{7 \cdot 3 \cdot 5}{317} = \frac{105}{317} 
.
\end{align*}
From the first row we get that:
\begin{align*}
 &  x_1 - \frac{2x_2}{5} + \frac{3x_3}{5} = -\frac{1}{5} \\
\implies & x_1 = \frac{2x_2}{5} - \frac{3x_3}{5} -\frac{1}{5} \\
\implies & 5x_1 = 2x_2 - 3x_3 -1 \\
\implies & 5x_1 = \frac{2 \cdot105}{317} + \frac{3 \cdot134}{317} -1 \\
& = \frac{210}{317} + \frac{402}{317} - 1 \\
& = \frac{210 + 402 - 317}{317} = \frac{295}{317} \\
\implies & x_1 = \frac{59}{317}
.
\end{align*}
So to summarise the solution is:
\begin{align*}
 &  x_1 = \frac{59}{317} \\
 &  x_2 = \frac{105}{317} \\
 &  x_3 = - \frac{134}{317}
.
\end{align*}


\end{solution}
\end{document}